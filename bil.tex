\documentclass[11pt]{article}
\usepackage{amsmath,amssymb}
\usepackage{supertabular}
\usepackage{geometry}
\usepackage{ifthen}
\usepackage{alltt}%hack
\geometry{a4paper,dvips,twoside,left=22.5mm,right=22.5mm,top=20mm,bottom=30mm}
\usepackage{color}

\begin{document}
\include{ott}

\title{A formal specification for BIL: BIL Instruction Language}
\maketitle

\tableofcontents
\clearpage

\section{Introduction}
\label{sec:intro}

This document describes the syntax and semantics of BAP Instruction
Language.  The language is used to represent a semantics of machine
instructions. Each machine instruction is represented by a BIL program
that tries to capture all side effect of the instruction.



\section{Syntax}
\label{sec:syntax}

\subsection{Metavariables}
\label{sec:meta}

We define a small set of metavariables that are used to denote string
and numeric literals and subscripts:

\ottmetavars

\subsection{BIL syntax}

BIL program is reperesented as a sequence of BIL statements. Each
statement performs some side-effectful computation.

\ottgrammartabular{
\ottbil\ottinterrule
}

\ottgrammartabular{
\ottstmt\ottinterrule
}

BIL expressions are side-effect free. Expressions include a usual set
of operations on bitvectors, like arithmetic operations and converting
bitvectors of one size to bitvectors of another size (casting in BIL
parlance).

\ottgrammartabular{
\ottexp\ottinterrule
\ottvar\ottinterrule
\ottbop\ottinterrule
\ottuop\ottinterrule
\ottendian\ottinterrule
\ottcast\ottinterrule
}

\ottgrammartabular{
\ottvar\ottinterrule
}

\ottgrammartabular{
\ottbop\ottinterrule
\ottuop\ottinterrule
\ottendian\ottinterrule
\ottcast\ottinterrule
}

The type system of BIL consists of two type families - immediate
values, indexed by a bitwidth, and storagies (aka memories), indexed
with address bitwidth and values bitwidth.

\ottgrammartabular{
\otttype\ottinterrule
}

\subsection{Bitvector syntax}
\label{sec:bitvector}

We represent concrete bitvector operations with the following syntax.
Operations marked with \verb|sbv| are signed. All other operations are
unsigned (if it does matter). Bitvector is represented by a pair of
value and size. Operations \verb|ext| and \verb|exts| performs
extract/extend operation. The former is unsigned (i.e., it extends
with zeros), the latter is signed. This operation extracts bits from a
bitvector starting from $\mathit{hi}$ and ending $\mathit{lo}$ bit
(both ends including). If $\mathit{hi}$ is greater than the bitwidth
of the bitvector, then it is extended with zeros (for \verb|ext|
operation) or with a sign bit (for \verb|exts|) operation.

\ottgrammartabular{
\ottword\ottinterrule
}

\subsection{Value syntax}
\label{sec:values}

Values are syntactic subset of expressions. They are used to represent
expressions that are not reducible.

We have three kinds of values --- immediates, represented as
bitvectors; unknown values and storages (aka memories) represented
symbolically as a list of assignments:

\ottgrammartabular{
\ottval\ottinterrule
}


\subsection{Formula syntax}
\label{sec:formula}

The following syntax is used to specify symbolic formulas in premises
of judgments.

We use $\Delta$ to denote set of bindings of variables to values. The
$\Delta$ context is represented as list of pairs. We also add a small
set of operations over natural numbers, like comparison and
arithmetics. Natural numbers are mostly used to reason about sizes of
bitvectors, that why they are often referred as $\mathit{sz}$.

We also add syntax for equality comparison for values and variables.


\ottgrammartabular{
\ottdelta\ottinterrule
}

\ottgrammartabular{
\ottformula\ottinterrule
}

\ottgrammartabular{
\ottnat\ottinterrule
}

\subsection{Instruction syntax}
\label{sec:insn}

To reason about the whole program we introduce a syntax for
instruction. An instruction is a binary string with length
$\mathit{w_2}$, that was read by a decoder from an address
$\mathit{w_1}$. The semantics of an instruction is described by a
$\mathit{bil}$ program.

\ottgrammartabular{
\ottinsn\ottinterrule
}

\section{Typing}
\label{sec:typing}

This section defines typing rules for BIL programs. Since BIL values
bears type information with them we do not need typing environment, so
the rules are fairly straightforward.

\ottdefnstypingXXstmt

\ottdefnstypingXXexp


\section{Operational semantics}

\subsection{Model of a program}

Program is coinductively defined as an infinite stream of program
states, produced by a step rule. Each state is represented with a
triplet $\Delta, w, var$, where $\Delta$ is a mapping from variables
to values, $w$ is a program counter, and $var$ is a variable
denoting currently active memory.

The \verb|step| rule defines how a machine instruction is
evaluated. We use ``magic'' function \verb|decode| that fetches
instruction from memory and decodes it to a BIL program.

A program counter is updated after each instruction.

\ottdefnsprogram


\end{document}
